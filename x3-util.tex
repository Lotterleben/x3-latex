%
% These are utility commands I often use in my documents.
% They're not loaded automatically by any of the templates, so you'll have to do that
% manually if you'd like to use them :)
%

% Needed for landscape figures
\usepackage{pdflscape}

\newcommand{\leadingzero}[1]{\ifnum #1<10 0\the#1\else\the#1\fi}
\newcommand{\todayInNumbers}{\leadingzero{\day}.\leadingzero{\month}.\the\year}

\newcommand{\mcal}[1]{\mathcal{#1}}

\newcommand{\op}[1]{\operatorname{#1}}

\ifDisableTodos
    \renewcommand{\todo}[1]{}
    \newcommand{\todoin}[1]{}
\else
    \newcommand{\todoin}[1]{\todo[inline]{#1}}
\fi

% Normal figure
\newcommand{\fig}[2]{
    \begin{figure}[h]
        \centering\footnotesize
        \includegraphics[width=\columnwidth]{figures/#1}
        \caption{#2}
        \label{fig:#1}
    \end{figure}
}

% Wide figure (for two-column mode)
\newcommand{\figw}[2]{
    \begin{figure*}[htbp]
        \centering\footnotesize
        \includegraphics[width=\textwidth]{figures/#1}
        \caption{#2}
        \label{fig:#1}
    \end{figure*}
}

\newcommand{\figwb}[2]{
    \begin{figure*}[hbp]
        \centering\footnotesize
        \includegraphics[width=\textwidth]{figures/#1}
        \caption{#2}
        \label{fig:#1}
    \end{figure*}
}

% Landscape figure
\newcommand{\figl}[2]{
    \begin{landscape}
         \begin{figure}
             \centering\footnotesize
             \includegraphics[width=\linewidth]{figures/#1}
             \caption{#2}
             \label{fig:#1}
         \end{figure}
    \end{landscape}
}
